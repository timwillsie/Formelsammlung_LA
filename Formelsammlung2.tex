% Mathe Formelsammlung für HM1 SoSe 2011
% 2 Seiten

% Dokumenteinstellungen
% ======================================================================	

% Dokumentklasse (Schriftgröße 6, DIN A4, Artikel)
\documentclass[6pt,a4paper]{scrartcl}

% Pakete laden
\usepackage[utf8]{inputenc}		% Zeichenkodierung: UTF-8 (für Umlaute)   
\usepackage[german]{babel}		% Deutsche Sprache
\usepackage{multicol}			% Spaltenpaket
\usepackage{amsmath}
\usepackage{amssymb}
\usepackage{esint}				% erweiterte Integralsymbole
\usepackage{multicol}			% ermöglicht Seitenspalten  
\usepackage{wasysym}			% Blitz
\usepackage{graphicx}
\usepackage[colorlinks=true, urlcolor=black]{hyperref}
  
% Seitenlayout und Ränder:
\usepackage{geometry}
\geometry{a4paper,landscape, left=6mm,right=6mm, top=0mm, bottom=3mm,includeheadfoot} 

%Kopf- und Fußzeile
\usepackage{fancyhdr}
\pagestyle{fancy}
\fancyhf{}

   \fancyfoot[C]{\textbf{Formelsammlung Lineare Algebra} von Tim Sievers}
   \renewcommand{\headrulewidth}{0.0pt} %obere Linie ausblenden
   \renewcommand{\footrulewidth}{0.1pt} %untere Linie

   \fancyfoot[R]{Stand: \today \qquad \thepage}
	
% Schriftart SANS für bessere Lesbarkeit bei kleiner Schrift
\renewcommand{\familydefault}{\sfdefault} 


% Custom Commands
\renewcommand{\thesubsection}{\arabic{subsection}}
\newcommand{\me}[1]{\ensuremath{\left\{#1\right\}}}
\newcommand{\dme}[2]{\ensuremath{\left\{#1\,\vert\,#2 \right\}}}
\newcommand{\abs}[1]{\ensuremath{\left\vert#1\right\vert}}
\newcommand{\un}[1]{\; \unit{#1} }
\newcommand{\unf}[2]{\;\left[ \unitfrac{#1}{#2} \right]}
\newcommand{\norm}[2][\relax]{\ifx#1\relax \ensuremath{\left\Vert#2\right\Vert}\else \ensuremath{\left\Vert#2\right\Vert_{#1}}\fi}
\newcommand{\enbrace}[1]{\ensuremath{\left(#1\right)}}
\newcommand{\nira}[1]{\ensuremath{\overset{n \rightarrow \infty}{\longrightarrow}}}
\newcommand{\os}[2]{\ensuremath{\overset{#1}{#2}}}
\makeatletter
\newcommand{\R}{\mathbb{R}}
\newcommand{\Ra}[0]{\ensuremath{\Rightarrow}}
\newcommand{\ra}[0]{\ensuremath{\rightarrow}}
\newcommand{\gk}[1]{\ensuremath{\left\lfloor#1\right\rfloor}}
\newcommand{\sprod}[2]{\ensuremath{%
  \setbox0=\hbox{\ensuremath{#2}}
  \dimen@\ht0
  \advance\dimen@ by \dp0
  \left\langle #1\rule[-\dp0]{0pt}{\dimen@},#2\right\rangle}}
  
%Custom functions
\DeclareMathOperator{\arccot}{arccot}
\DeclareMathOperator{\Bild}{Bild}
\DeclareMathOperator{\diag}{diag}
\DeclareMathOperator{\rg}{rg}


% Dokumentbeginn
% ======================================================================
\begin{document}
%\section{}
% ----------------------------------------------------------------------

% Aufteilung in Spalten
\begin{multicols*}{3}

\subsection{Aussagenlogik}
\subsubsection{Gesetze der Aussagenlogik}
\begin{tabular}{l|l|l}	
	\textbf{Idempotenz} 	& $A \land A \; \equiv \; A $ & $A \lor A \; \equiv \; A$ \\
	\hline
	\textbf{Kommutativ}	& $A \land B \; \equiv \; B \land A$ & $A \lor B \; \equiv \; B \lor A$ \\
	\hline
	\textbf{Assoziativ} 	& \multicolumn{2}{l}{$(A \land B)\land C \; \equiv \; A \land (B \land C)\; \equiv \; A \land B \land C$} \\
						& \multicolumn{2}{l}{$(A \lor B) \lor C \; \equiv \; A \lor (B \lor C)\; \equiv \; A \lor B \lor C$}\\
	\hline
	\textbf{Absorption} 	& $A \lor (A \land B) \; \equiv \; A$ & $A \land (A \lor B) \; \equiv \; A$ \\
						& $A \land (\neg A \lor B) \; \equiv \; A \land B$ & $A \lor (\neg A \land B) \; \equiv \; A \lor B$ \\
						& $\neg A \land (A \lor B) \; \equiv \; \neg A \land B$ & $\neg A \lor (A \land B) \; \equiv \; \neg A \lor B$ \\
						& $A \land (B \lor \neg B) \; \equiv \; A$ & $A \lor (B \land \neg B) \; \equiv \; A$ \\
	\hline
	\textbf{Neutralität}	& $A \land 1 \; \equiv \; A$ & $A \lor 0 \; \equiv \; A$ \\
						& $A \land 0 \; \equiv \; 0$ & $A \lor 1 \; \equiv \; 1$ \\
	\hline
	\textbf{Distributiv}	& \multicolumn{2}{l}{$A \land (B \lor C) \; \equiv \; (A \land B) \lor (A \land C)$} \\
						& \multicolumn{2}{l}{$A \lor (B \land C) \; \equiv \; (A \lor B) \land (A \lor C)$} \\
	\hline
	\textbf{Komplementär}	& $A \land \neg A \; \equiv \; 0$ & $A \lor \neg A \; \equiv \; 1$ \\
	\hline
	\textbf{De Morgan}	& $\neg (A \land B) \; \equiv \; \neg A \lor \neg B$ & $\neg (A \lor B) \; \equiv \; \neg A \land \neg B$ \\
	\hline
	\textbf{Implikation}	& $A \rightarrow B \; \equiv \; \neg A \lor B$ & $A \rightarrow B \; \equiv \; \neg B \rightarrow \neg A$ \\
	\hline
	\textbf{Äquivalenz}	& \multicolumn{2}{l}{$A \leftrightarrow B \; \equiv \; (A \rightarrow B) \land (B \rightarrow A) \; \equiv \; (\neg A \lor B) \land (\neg B \lor A) $} \\	
\end{tabular}

\subsubsection{Bindungsstärke (stark $\rightarrow$ schwach)}
$\neg, \land, \lor, \rightarrow, \leftrightarrow$

\subsubsection{Wahrheitstabellen}
\begin{tabular}{cc}
	\begin{tabular}{cc|c}
	A & B & $A \rightarrow B$ \\
	\hline
	0 & 0 & 1 \\
	0 & 1 & 1 \\
	1 & 0 & 0 \\
	1 & 1 & 1 \\
	\end{tabular}
	&
	\begin{tabular}{cc|c}
	A & B & $A \leftrightarrow B$ \\
	\hline
	0 & 0 & 1 \\
	0 & 1 & 0 \\
	1 & 0 & 0 \\
	1 & 1 & 1 \\
	\end{tabular}
	
\end{tabular}
\subsubsection{Hinweise zur Implikation}
$A \rightarrow B$ bedeutet ``Wenn A gilt, dann gilt B'.' bzw. ``Nur wenn B gilt, dann gilt A.'' \\
Aus einer wahren Aussage kann keine falsche Aussage folgen.
\\TODO: Beispielformulierung aus den Textaufgaben einfügen

\subsubsection{Begriffe}
\textbf{Tautologie} ist eine Formel, die stets wahr ist (unabhängig von den Wahrheitswerten ihrer Aussagenvariablen). \\
\textbf{Kontradiktion} ist eine Formel die stets falsch ist. \\
\textbf{logisch äquivalent} sind zwei Formeln, wenn sie in jeder Zeile der Wahrheitstafel jeweils die selben Werte haben. \\

\subsubsection{Normalformen}
\textbf{Disjunktive Normalform (DNF)} Disjunktion von Konjunktionen von Literalen \\
\textbf{Erstellung:} Alle Zeilen der Wahrheitstabelle betrachten, die wahr sind (1). Die Literale dieser Zeilen 1:1 übernehmen \\
Beispiel:
\begin{tabular}{l|l|l|l}
A & B & F & DNF-Therm \\
\hline
0 & 1 & 1 & $\neg A \land B$ \\
1 & 0 & 1 & $A \land \neg B$ \\
\end{tabular} \\
ergibt $(\neg A \land B) \lor (A \land \neg B)$ \\
\\
\textbf{Konjuktive Normalform (KNF)} Konjunktion von Disjunktionen von Literalen\\
\textbf{Erstellung:} Alle Zeilen betrachten, die unwahr (0) sind. Die Literale dieser Zeilen negieren. \\
Beispiel:
\begin{tabular}{l|l|l|l}
A & B & F & KNF-Therm \\
\hline
0 & 1 & 0 & $A \lor \neg B$ \\
1 & 0 & 1 & $\neg A \lor B$ \\
\end{tabular} \\
ergibt $(A \lor \neg B) \land (\neg A \lor B)$ \\
\\
\textbf{Kanonische Normalform (kanonische NF)}\\
Jeder Klammerausdruck in der Normalform enthält sämtliche Aussagevariablen. – Es gibt nur eine einzige
kanonische DNF und nur eine einzige kanonische KNF zur Realisierung einer booleschen Funktion. 


\subsection{Matrizen}
\subsubsection{Rechenoperationen}
\textbf{Addition / Subtraktion} \\
Elementweise addieren.\\
Beispiel:\\
$A=\begin{pmatrix}1&2&3\\4&5&6\end{pmatrix}, B=\begin{pmatrix}5&-1&2\\0&2&-3\end{pmatrix}$ \\
ergibt $A+B=\begin{pmatrix}6&1&5\\4&7&3\end{pmatrix},	A-B=\begin{pmatrix}-4&3&1\\4&3&9\end{pmatrix}$ \\
\\
\textbf{skalare Multiplikation} \\
Jedes Element mit dem Skalar multiplizieren \\
Beispiel: \\
$-2A = \begin{pmatrix}-2&-4&-6\\-8&-10&-12\end{pmatrix}$ \\
\\
\textbf{Rechengesetze}
\begin{tabular}{l|l}
A + B = B + A & Kommutationgesetz \\
(A + B) + C = A + (B + C) = A + B + C & Assoziativgesetz \\
A + \textbf{0} (``Nullmatrix'') = A & neutrales Element \\
A + (-A) = \textbf{0} & \\
$1 \cdot A$ = A & \\
$0 \cdot A$ = \textbf{0} & \\
$\lambda \cdot (\mu \cdot$  A) = $(\lambda \cdot \mu) \cdot $ A & \\
$\lambda\cdot(A\pm B) = (\lambda\cdot A) \pm (\lambda\cdot B)$  &  \\
$(\lambda\pm \mu)\cdot A = (\lambda\cdot A) \pm (\mu\cdot A)$  & \\
\end{tabular} \\
\\
\subsubsection{Transponieren}
Eine Matrix transponieren heißt, ihre Zeilen und Spalten vertauschen. Genauer gesagt, die Zeilen werden zu Spalten (und umgekehrt). Aus einer $m{\times}n$-Matrix A wird dabei eine $n{\times}m$-Matrix $A^\top$\\
\textbf{Kurz gesagt}: Spalten werden Zeilen, Zeilen werden Spalten. Symmetrische Matrizen werden entlang der Diagonalen gespiegelt.\\
Beispiele:\\
$A=\begin{pmatrix}1&2&3\\4&5&6\end{pmatrix}, B=\begin{pmatrix}5&-1&2\end{pmatrix}, C=\begin{pmatrix}3\\2\\0\end{pmatrix}$ \\
ergibt	 \\
$A^\top=\begin{pmatrix}1&4\\2&5\\3&6\end{pmatrix}	B^\top=\begin{pmatrix}5\\-1\\2\end{pmatrix}	C^\top=\begin{pmatrix}3&2&0\end{pmatrix}$ \\
\\
Es gilt stets: $(A^\top)^\top = A$ \\
\\
\subsubsection{Skalarprodukt}
Seien 
$v = \begin{pmatrix}a_1\\a_2\\\vdots\\a_n\end{pmatrix}, \quad w = \begin{pmatrix}b_1\\b_2\\\vdots\\b_n\end{pmatrix} \quad \in \R^n$ \\
Das Skalarprodukt von v und w ist wie folgt definiert:

$v \cdot w = a_1 b_1 + ... +a_n b_n$
\\
\subsubsection{Matrixmultiplikation}
Sei   $A \in \R^{m\times n}$   und   $B \in \R^{n\times k}.$\\
Das Produkt $C := A\cdot B \in \R^{m\times k}$ ist definiert durch\\
$c_{ij} = a_{i*} \cdot b_{*j}.$\\
\\
\textbf{Umgangssprachlich:} Zeile 1 von Matrix A * Spalte 1 von Matrix B ergibt Element 1. Spalte/1. Zeile der Ergebnismatrix\\
\\
\textbf{Wichtig:} Die Multiplikation ist nur zwischen Matrizen definiert, wenn die Spaltenanzahl der ersten der Zeilenanzahl der zweiten entspricht\\
\textbf{die Dimension der Ergebnismatrix:} $(m\times n)\cdot (n \times k) = (m \times k)$ \\
\subsubsection{Rechenregeln Matrixmultiplikation}
\begin{tabular}{l}
$(A \cdot B) \cdot C = A \cdot (B \cdot C)  = A \cdot B \cdot C$ \\
$(\lambda A) \cdot B = A \cdot (\lambda B)  = \lambda (A \cdot B)$ \\
$A \cdot e_j = a_{*j}$ \\
$e_i^\top \cdot A = a_{i*}$ \\
$A \cdot E = E \cdot A  = A$ \\
$A \cdot \textbf{0} = \textbf{0} \cdot A  = $\textbf{0} \\
$A \cdot(B + C) = A \cdot B + A \cdot C $ 
$(A+B) \cdot C = A \cdot C + B \cdot C $ \\
$(A \cdot B)^\top = B^\top \cdot A^\top $ \\
\end{tabular}

\subsubsection{Determinanten}
\textbf{2$\times$2 Matrix:} $\begin{vmatrix}
a & b \\
b & c
\end{vmatrix} = a \cdot d - b \cdot c$
\\
\\
\\
\textbf{3$\times$3 Matrix:} $
\begin{vmatrix}
a_11 & a_12 & a_13 \\
a_21 & a_22 & a_23 \\
a_31 & a_32 & a_33 \\
\end{vmatrix}
=
\begin{vmatrix}
a_11 & a_12 & a_13 \\
a_21 & a_22 & a_23 \\
a_31 & a_32 & a_33 \\
\end{vmatrix}
\begin{matrix}
a_11 & a_12 \\
a_21 & a_22 \\
a_31 & a_32 \\
\end{matrix}
\\
\\
= a11 \cdot a22 \cdot a33 + a12 \cdot a23 \cdot a31 + a13 \cdot a21 \cdot a32 - a31 \cdot a22 \cdot a13 - a32 \cdot a23 \cdot a11 - a33 \cdot a21 \cdot a12
$
\\
\\
\textbf{Umgangssprachlich:} Die Hauptdiagonalen multiplizieren und addieren, dann die Nebendiagonalen multiplizieren und subtrahieren.
\\
\\
\textbf{La Place Entwicklungssatz:} Suche die Zeile/Spalte mit den meisten Nullen. Für jedes Element ungleich 0 berechne die umliegende Determinante und addiere/subtrahiere diese.\\
Ob addiert oder subtrahiert wird hängt von der Position ab. Positiv und Negativ wechselt sich in der Matrix ab:\\
$
\begin{vmatrix}
+ & - & + & - \\
- & + & - & + \\
+ & - & + & - \\
- & + & - & +
\end{vmatrix}
$
\subsubsection{Matrix invertieren}
Eine Matrix A ist nur dann invertierbar, wenn gilt $det(A) \neq 0$\\
\\
Die Inverse lässt sich mit dem Gauß-Algorithmus berechnen. Linke Seite ist die Matrix A und rechts steht die Einheitsmatrix der jeweiligen Dimension. \\
Nun formt man das LGS solange um, bis \\
1. links die Einheitsmatrix übrig bleibt, dann kann man rechts die Inverse ablesen \\
2. links eine Nullzeile entsteht (rechts kann dies nicht passieren!). In diesem Fall ist die Matrix A nicht invertierbar.\\
\\
\end{multicols*}
% Ende der Spalten


% Dokumentende
% ======================================================================
\end{document}